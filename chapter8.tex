\chapter{辅助}

如今有很多辅助工具可以帮助我们更好地完成\LaTeX{}文档.

\section{MathJax}

有些``所见即所得''MarkDown编辑器能很好地配合MathJax显示\LaTeX{}公式.
平时可以先用此类软件写一些片段, 若能正确展示, 再复制到\TeX{}文件中, 这样文本编写效率会更高.
实际上, 如今Visual Studio Code内置的MarkDown预览功能已经相当强大(使用的KaTeX与MathJax略有区别),
而且是免费的.
之前我们推荐过Typora(\url{https://typora.io})或Mark Text(\url{https://github.com/marktext/marktext}), 大家也可以使用.

\section{公式识别}

目前能够识别公式并转换成\LaTeX{}源代码的工具很多, 其功能也在不断完善. 在网络上随手一搜就能找到.

\CJKsout*[format=\color{red}]{Microsoft Math Solver(``微软数学'')这款APP的主要功能是求解数学问题, 但是我们可以用来处理复杂的公式, 在平板上手写识别后可以复制\LaTeX{}源代码(识别能力只能说是勉强可用). 不过很遗憾, 这款APP已经下架.}

\section{表格自动转换}

对于数据量较大的表格, 可以寻找转换工具(也有很多在线版本), 处理起来会节约很多时间.

另外, 实验数据最好由编程语言提供的软件包转换, 如果能获得\texttt{csv}文件可用\texttt{csvsimple}宏包自动根据该文件生成表格.

\section{PDF代换法}

如果使用其他软件绘制的图表, 可以转换成PDF文件再以图片形式统一包含, 这样可以节约不少时间.
例如:
\begin{Verbatim}
\begin{table}[h]
    \centering
    \includegraphics[width=.8\textwidth]{table.pdf}
    \caption{形为表格实为图片}
\end{table}
\end{Verbatim}
需要注意的是, PDF需要裁剪成合适的尺寸(macOS可以使用自带的Preview完成).
对于Word而言, 可以为指定的表格设置合适的纸张大小以及边距, 这样导出的PDF文件可直接使用.
TeX Live还提供了\texttt{pdfcrop}工具(命令行方式)来裁白边,注意MacTeX将其安装在\texttt{/Library/TeX/texbin}目录下.
