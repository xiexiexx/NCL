\chapter{辅助}

如今有很多辅助工具可以帮助我们更好地完成\LaTeX{}文档.

\section{MathJax}

有些``所见即所得''MarkDown编辑器能很好地配合MathJax显示\LaTeX{}公式.
平时可以先用此类软件写一些片段, 若能正确展示, 再复制到\TeX{}文件中, 这样文本编写效率会更高.
例如Typora(\url{https://typora.io})和Mark Text(\url{https://marktext.app})

\section{Microsoft Math Solver}

Microsoft Math Solver(``微软数学'')这款APP的主要功能是求解数学问题, 但是我们可以用来处理复杂的公式, 在平板上手写识别后可以复制\LaTeX{}源代码.

\section{表格自动转换}

对于数据量较大的表格, 可以寻找转换工具(也有很多在线版本), 会节约很多时间.

另外, 实验数据最好由编程语言提供的软件包转换.

\section{PDF代换法}

如果使用其他软件绘制的图表, 可以转换成PDF文件再以图片形式统一包含, 这样可以节约不少时间.
例如:
\begin{Verbatim}
\begin{table}[h]
    \centering
    \includegraphics[width=.8\textwidth]{table.pdf}
    \caption{形为表格实为图片}
\end{table}
\end{Verbatim}
