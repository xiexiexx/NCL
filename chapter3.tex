\chapter{公式}

  最常见的是行内公式, 例如集合$X$中的元素$a$, 初学者容易直接使用普通符号而不是数学模式.
另外一个问题是斜体, 最好使用\emph{斜体}方式.

  行间公式一般带编号, 例如公式\eqref{eq:log_factorial}给出了渐近记号.
\begin{equation}
\label{eq:log_factorial}
\log(n!) = \Theta(n\log{n})
\end{equation}

  注意写一段就编译一段, 这样容易查错, 特别是公式.

\section{初学\LaTeX{}常见的数学符号误用}

  我们给出几个典型例子(括号内附有代码).
\begin{itemize}
  \item 是$\log{x}$(\verb@$\log{}x$@)而不是$log x$(\verb@$log x$@);
  \item 是$\min$(\verb@$\min$@)和$\max$(\verb@$\max$@)而不是$min$(\verb@$min$@)和$max$(\verb@$max$@);
  \item 是$\Pr$(\verb@$\Pr$@)而不是$Pr$(\verb@$Pr$@);
  \item 是$\sin{x}$(\verb@$\sin{}x$@)和$\cos{x}$(\verb@$\cos{}x$@)而不是$sin{x}$(\verb@$sin x$@)和$cos x$(\verb@$cos{x}$@);
  \item 是$x \times y$(\verb@$x \times y$@)而不是$x * y$(\verb@$x * y$@);
  \item 是$\arg$(\verb@$\arg$@)而不是$arg$(\verb@$arg$@);
\end{itemize}
