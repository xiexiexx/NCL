\chapter{表格}


  一般采用浮动的表格, 以选项\texttt{[!htbp]}标记, 注意表格不能写``如下所示''
要写``如\ref{table:trilogy}所示''.


  表\ref{tab:trilogy}使用了单元格不同位置的标记(例如\texttt{c}/\texttt{l}/\texttt{r}标记).

\begin{table}[!hbtp]
\centering
\begin{tabular}{|c|l|r|}
\hline
    & Books                         &     \\ \hline
1   & Introduction to Algorithms    & 3 \\
2   & The Algorithm Design Manual   & 2 \\
3   & Algorithms                    & 4 \\
\hline
\end{tabular}
\caption{算法三部曲}
\label{tab:trilogy}
\end{table}


  表\ref{tab:number}下面还有子表\ref{tab:number:d}和\ref{tab:number:b}, 注意命名不同.

\begin{table}[!htb]
\centering
  \caption{主表}
  \label{tab:number}
\begin{subtable}[t]{2in}
  \centering
  \begin{tabular}{|l|l|l|l|}
  \hline
  0 & 1 & 2 & 3 \\
  \hline
  4 & 5 & 6 & 7 \\
  \hline
  \end{tabular}
  \caption{子表(十进制)}\label{tab:number:d}
\end{subtable}
\quad
\begin{subtable}[t]{2in}
  \centering
  \begin{tabular}{|l|l|l|l|}
  \hline
  000 & 001 & 010 & 011 \\
  \hline
  100 & 101 & 110 & 111 \\
  \hline
  \end{tabular}
  \caption{子表(二进制)}\label{tab:number:b}
\end{subtable}
\end{table}

最好要把每一页填满, 这样排版问题会少很多.

\begin{table}[h]
    \centering
    \includegraphics[width=.8\textwidth]{books.jpg}
    \caption{算法三部曲}
\end{table}
