\chapter{引用}

最常见的引用是参考文献的引用. 例如参考文献\cite{SX}的标记为\verb=SX=(也即\verb=\bibitem{SX}=), 
于是我们可以使用\verb=\cite{SX}=实现引用参考文献\cite{SX}的效果, 
而这本书便是使用\LaTeX{}排版. 

引用需要排次序, 所以需要两次编译, 第1次识别所有的标记并编号, 第2次将编号填入并显示. 

其他引用一般使用\verb=label=给出标记, 再用\verb=ref=命令引用. 
例如以\verb=\label{eq:log_factorial}=给出公式标记\verb=eq:log_factorial=, 
再用\verb=\ref{eq:log_factorial}=引用得到\ref{eq:log_factorial}. 
为了区别不同的标记, 可用前缀配合冒号区分: 公式的前缀\verb=eq=, 图的前缀\verb=fig=, 表的前缀\verb=tab=.
列举如下:
\begin{itemize}
    \item 公式标记为\verb=eq:log_factorial=, 也即公式\ref{eq:log_factorial}.
    \item 图标记为\verb=fig:books=, 也即图\ref{fig:books}.
    \item 表标记为\verb=tab:trilogy=, 也即表\ref{tab:trilogy}.
\end{itemize}
