\chapter{图片}

  一般采用浮动的图, 以选项\texttt{[!htbp]}标记, 如图\ref{fig:algorithm}所示,
注意其中还有两个子图\ref{fig:algorithm:a}和\ref{fig:algorithm:b}.

\begin{figure}[!htbp]
\centering
\begin{subfigure}[t]{.3\textwidth}
  \centering
  \includegraphics[width=.95\textwidth]{books.jpg}
  \caption{算法}
  \label{fig:algorithm:a}
\end{subfigure}
\quad
\begin{subfigure}[t]{.3\textwidth}
  \centering
  \includegraphics[width=.95\textwidth]{books.jpg}
  \caption{还是算法}
  \label{fig:algorithm:b}
\end{subfigure}
\caption{主图}
\label{fig:algorithm}
\end{figure}


  如果有特殊的需要, 可以用固定位置的图片, 在后面加上\texttt{[h]}选项即可, 例如图\ref{fig:books}。

\begin{figure}[h]
    \centering
    \includegraphics[width=.8\textwidth]{books.jpg}
    \caption{算法三部曲}
    \label{fig:books}
\end{figure}

  当然也不一定能够完全在当前位置, 可能当前位置不够会挤到下一页.

  不要用屏幕截图, 实验结果可以用软件的导出图,%
\footnote{得用PNG格式, 其他的展示型图像可以用JPG格式.}
数据可以导出文本利用表格或者抄录形式.
